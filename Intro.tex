	The properties of porous materials are important in acoustical engineering. Porosity (the 
ratio of pore volume to total volume) and acoustic impedance factor into the sound damping 
capabilities of materials. NASA has a vested interest in this field. NASA monitors the acoustical 
environment on every manned spacecraft (i.e. the ISS, Orion Capsule). The Johnson Space Center 
(JSC) Acoustics and Noise Control Laboratory (ANCL) is tasked with ensuring a "Safe, Healthy, and 
Habitable Vehicle Acoustic Environment" (\cite{nasa_acoustics}). This laboratory uses a number of 
techniques to measure the acoustic properties of materials used to control noise in a manned 
spacecraft. This could be the noise from a cooling fan on a piece of equipment or duct noise from
air circulation. One of the measurement instruments used at ANCL is an acoustic impedance tube.
	When designing these materials for noise control, it is important to account 
for how much airflow a system may need through the sound impeding material. To account for this,
the airflow resistance of the materials can be studied. Airflow resistance is defined as the 
differential pressure over the material divided by the volumetric airflow rate through the 
material(\cite{liu2018numerical}).
\begin{equation} \label{Airflow Resistance} 
R= \frac{\Delta p}{q_v}
\end{equation}
The volumetric airflow rate is determined by the air speed through the material multiplied by the 
cross sectional area of the output space where speed is being measured.
\begin{equation}
q_v = v \pi r^2 \quad \text{where } r = 0.0330\ \mathrm{m}
\end{equation}
\begin{equation}
q_v = v(0.00342 \mathrm{m^2})
\end{equation}
Comparing how material properties affect both sound impedance and airflow resistance will allow for 
the optimization of acoustic materials. However, this project will solely focus on the airflow 
resistance of these materials. 

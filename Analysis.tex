	Using the average differential pressure for each sample and the readings from the 
flow meter, the airflow resistance can be calculated. Unfortunately, the speeds at which the 
device is being run are in the lower limits that the flow meter can detect, so the flow meter 
cannot stabilize and fluctuates in its flow speed reading. What can be found from this is a 
range of airflow resistance values. For the 6 cm sample(Figure \ref{fig:6cm_dp_graph}), the 
average differential pressure was 13.635 Pa and the range of flow speeds was 0.30 to 0.46 
m/s. The lower limit of the airflow resistance values was determined using Equation 
\ref{Airflow Resistance}. 

\begin{align}
R &= \frac{13.635\ \mathrm{Pa}}{0.46\ \mathrm{m/s} \times 0.00342} \\
  &= 8.7 \times 10^3\ \mathrm{Pa\cdot s/m^3}\\
\end{align}
And the higher limit:
\begin{align}
R &= \frac{13.635\ \mathrm{Pa}}{0.30\ \mathrm{m/s} \times 0.00342} \\
  &= 1.3 \times 10^4\ \mathrm{Pa\cdot s/m^3} 
\end{align}

This gives an airflow resistance range from $8.7 \times 10^3$ to $1.3 \times 10^4\ 
\mathrm{Pa\cdot s/m^3}$ for the 6 cm sample.
	The same method was applied to the 9 cm(Figure \ref{fig:9cm_dp_graph}) and 12 
cm(Figure \ref{fig:12cm_dp_graph}) samples. The average differential pressure for the 9 cm 
sample was 13.170 Pa, and 17.055 Pa for teh 12 cm sample. The speed range was 0.25 to 0.42 
m/s for both the 9 and 12 cm samples. The airflow resistance ranges for the samples were $9.2 
\times 10^3$ to $1.5 \times 10^4\ \mathrm{Pa\cdot s/m^3}$ and $1.2 \times 10^4$ to $2.0 
\times 10^4\ \mathrm{Pa\cdot s/m^3}$ respectively.

 
